%%%%%%%%%%%%%%%%%%%%%%%%%%%%%%%%%%%%%%%%%
% Journal Article
% LaTeX Template
% Version 1.4 (15/5/16)
%
% This template has been downloaded from:
% http://www.LaTeXTemplates.com
%
% Original author:
% Frits Wenneker (http://www.howtotex.com) with extensive modifications by
% Vel (vel@LaTeXTemplates.com)
%
% License:
% CC BY-NC-SA 3.0 (http://creativecommons.org/licenses/by-nc-sa/3.0/)
%
%%%%%%%%%%%%%%%%%%%%%%%%%%%%%%%%%%%%%%%%%

%----------------------------------------------------------------------------------------
%	PACKAGES AND OTHER DOCUMENT CONFIGURATIONS
%----------------------------------------------------------------------------------------

\documentclass[twoside,twocolumn]{article}



\usepackage{blindtext} % Package to generate dummy text throughout this template 

\usepackage[sc]{mathpazo} % Use the Palatino font
\usepackage[T1]{fontenc} % Use 8-bit encoding that has 256 glyphs
\linespread{1.05} % Line spacing - Palatino needs more space between lines
\usepackage{microtype} % Slightly tweak font spacing for aesthetics

\usepackage[english]{babel} % Language hyphenation and typographical rules

\usepackage[hmarginratio=1:1,top=32mm,columnsep=20pt]{geometry} % Document margins
\usepackage[hang, small,labelfont=bf,up,textfont=it,up,figurename=Obrázek,tablename=Tabulka]{caption} % Custom captions under/above floats in tables or figures
\usepackage{booktabs} % Horizontal rules in tables

\usepackage{lettrine} % The lettrine is the first enlarged letter at the beginning of the text

\usepackage{enumitem} % Customized lists
\setlist[itemize]{noitemsep} % Make itemize lists more compact

\usepackage{abstract} % Allows abstract customization
\renewcommand{\abstractnamefont}{\normalfont\bfseries} % Set the "Abstract" text to bold
\renewcommand{\abstracttextfont}{\normalfont\small\itshape} % Set the abstract itself to small italic text

\usepackage{titlesec} % Allows customization of titles
\renewcommand\thesection{\Roman{section}} % Roman numerals for the sections
\renewcommand\thesubsection{\roman{subsection}} % roman numerals for subsections
\titleformat{\section}[block]{\large\scshape\centering}{\thesection.}{1em}{} % Change the look of the section titles
\titleformat{\subsection}[block]{\large}{\thesubsection.}{1em}{} % Change the look of the section titles

%\usepackage{fancyhdr} % Headers and footers
%\pagestyle{fancy} % All pages have headers and footers
%\fancyhead{} % Blank out the default header
%\fancyfoot{} % Blank out the default footer
%\fancyhead[C]{} % Custom header text
%\fancyfoot[RO,LE]{\thepage} % Custom footer text

\usepackage{titling} % Customizing the title section

\usepackage{hyperref}
\usepackage{graphicx} % For hyperlinks in the PDF


%----------------------------------------------------------------------------------------
%	TITLE SECTION
%----------------------------------------------------------------------------------------

\setlength{\droptitle}{-4\baselineskip} % Move the title up

\pretitle{\begin{center}\Huge\bfseries} % Article title formatting
\posttitle{\end{center}} % Article title closing formatting
\title{Heuristika simulovaného ochlazování pro řešení MaxWeightedSAT} % Article title
\author{%
    \textsc{Luboš Zápotočný}\\[1ex] % Your name
    \normalsize České vysoké učení technické v Praze - fakulta informačních technologií \\ % Your institution
    \normalsize \href{mailto:zapotlub@fit.cvut.cz}{zapotlub@fit.cvut.cz} % Your email address
%\and % Uncomment if 2 authors are required, duplicate these 4 lines if more
%\textsc{Jane Smith}\thanks{Corresponding author} \\[1ex] % Second author's name
%\normalsize University of Utah \\ % Second author's institution
%\normalsize \href{mailto:jane@smith.com}{jane@smith.com} % Second author's email address
}
\date{} % Leave empty to omit a date \today
%\renewcommand{\maketitlehookd}{%
%\begin{abstract}
%\noindent \blindtext % Dummy abstract text - replace \blindtext with your abstract text
%\end{abstract}
%}

%----------------------------------------------------------------------------------------

\begin{document}

% Print the title
    \maketitle

%----------------------------------------------------------------------------------------
%	ARTICLE CONTENTS
%----------------------------------------------------------------------------------------


    \section{Úvod}

    Problém splnitelnosti booleovské formule (ozačováno z angličtiny SATISFIABILITY, zkráceně SAT) označuje problém
    nalezení splňujícího (vyhovujícího) ohodnocení logické formule v~konjunktivní normálním formě tak, aby byly všechny její
    klauzule splněné.

    SAT byl první problém o kterém se dokázalo, že je NP-úplný~\cite{CookLevin1971}.
    Tedy pro tento problém neexistuje (za předpokladu P $\neq$ NP) efektivní algoritmus, který by tento problém řešil v polynomiálním čase.
    Jelikož se jedná o NP-těžký (NP-úplné problémy jsou podmnožinou NP-těžkých) problém, lze na tento problém převést instance všech problému ze tříd P a NP.

    MaxWeightedSAT označuje optimalizační verzi hledání SAT ohodnocení zároveň s~kritériem pro maximalizaci součtu
    vah proměnných ohodnocených 1 (True).
    Problém je tedy rozšířen o atributy $w(x_i)$ pro všechny proměnné $x_i$ reprezentující váhové ohodnocení jednotlivých
    proměnných.

    Tato práce se zaměřuje na řešení výše zmíněného NP-těžkého optimzačního problému pomocí heuristiky simulovaného ochlazování.

    Algoritmus náhodně prochází stavovým prostorem ohodnocení proměnných formule tak, aby maximalizoval součet vah pozitivně
    ohodnocených proměnných a zároveň aby toto ohodnocení splňovanou zadanou formuli.

    Stavový prostor je tedy vektor (pole) boolevských ohodnocení (True/False) jednotlivých proměnných.
    Operátorem přechodu do nového stavu je logická změna jednoho náhodného bitu v tomto vektoru.

    Algoritmus přechází do zlepšujících stavů a s určitou pravděpodobností přechází také do zhrošujících stavů.
    Tímto postupem se heuristika snaží zamezit uvíznutí v lokálních extrémech.

    Heuristika začíná s vysokou teplotou, která ovlivňuje mimo jiné také pravděpodobnost přijetí nezlepšujících stavů.
    To vede k velkému prozkoumání stavového prostoru v prvních krocích algoirtmu.
    Každý krok algorimu sníží tuto teplotu o násobek chladícího faktoru (lineárně).
    Tímto chlazením se také snižuje pravděpodobnost přijetí zhrošujícího stavu a~heuristika tímto konverguje k nalezení
    optimálního řešení.

    Podrobný popis datových sad lze najít na stránkách kurzu NI-KOP~ČVUT~FIT~\cite{coursesData}.

%------------------------------------------------


    \section{Implementace}

    Heuristika a experimentální vyhodnocení byly naprogramovány v jazyce Python.
    Při hledání ideálních parametrů heuristiky či efektivity algoritmu není uvažována efektivita jazyka jako takového.
    Všechny metriky jsou univerzálně přenositelné mezi různými hardwarovými platformami.

    Hlavní část heuristiky je zobrazena na obrázku~\ref{fig:main-loop} kde lze nahlédnou, jak algoritmus pracuje.

    \begin{figure}
        \centering
        \includegraphics[width=7cm]{images/main-loop}
        \caption{Hlavní smyčka algoritmu}
        \label{fig:main-loop}
    \end{figure}

    Pravděpodobnost přijetí zhoršujícího řešení je závislá na rozdílu hodnot účelové funkce mezi aktuálním řešením a potenciálním novým řešením.
    Zároveň je závislá na aktuální teplotě. Čím větší teplota, tím větší pravděpodobnost, že bude přijato zhoršující řešení.
    Konkrétní implementaci lze vidět na obrázku~\ref{fig:probabilty}.
    Výsledek této funkce je následně v hlavním cyklu porovnán s náhodnou hodnotou v internalu od 0 do 1 a pokud tato hodnota
    převyšuje vygenerovanou náhodnou hodnotu přijímáme zhrošující řešení.

    Zlepšující řešení přijímáme vždy.

    \begin{figure}
        \centering
        \includegraphics[width=7cm]{images/probabilty}
        \caption{Výpočet pravděpodobnosti přijetí zhoršujícího řešení}
        \label{fig:probabilty}
    \end{figure}

    Průchod stavovým prostorem zajišťuje funkce \emph{perturb\_solution} (obrázek~\ref{fig:perturb}) která flipuje náhodné bity v aktuálním ohodnocení.
    Tímto se můžeme dostat k méně optimálnímu řešení než které aktuálně máme, ale zajišťujeme si tím větší průchod
    možných konfigurací a mitigujeme uváznutí v lokálních extrémech.

    \begin{figure}
        \centering
        \includegraphics[width=7cm]{images/perturb}
        \caption{Perturbace ohodnocení proměnných}
        \label{fig:perturb}
    \end{figure}

    Algoritmus simulovaného ochlazování postupně odhlazuje aktuální teplotu.
    Na konci každé iterace je v hlavním cyklu volána metoda zobrazená na obrázku~\ref{fig:cool}.
    Chladící faktor je parametr heuristiky v rozsahu od 0 do 1, který je v následujících sekcích práce experimentálně nastaven
    na ideální hodnotu.

    \begin{figure}
        \centering
        \includegraphics[width=7cm]{images/cool}
        \caption{Metoda pro chlazení teploty}
        \label{fig:cool}
    \end{figure}

    Při implementaci a testování heuristiky byly použity tyto testovací sady

    \begin{itemize}
        \item wuf20-71
        \item wuf20-71R
        \item wuf20-91
        \item wuf20-91R
        \item wuf50-200
        \item wuf50-219
        \item wuf50-218R
        \item wuf75-325
        \item wuf100-430
    \end{itemize}

    Parametry heuristiky jsou následující

    \begin{itemize}
        \item \emph{initial\_temperature} - počáteční teplota
        \item \emph{final\_temperature} - konečná teplota
        \item \emph{num\_iterations\_per\_temperature} - počet iterací vnitřního cyklu ochlazování
        \item \emph{perturbation\_flips} - počet náhodně vybraných bitů pro flipnutí při prohledávání stavového prostoru
        \item \emph{cooling\_factor} - desetinné číslo (0, 1) reprezentující lineární funkci pro chlazení
        \item \emph{penalty} - celočíselná (záporná) hodnota penalizující nesplněnou klauzi (\emph{None} pro aktivaci adaptivní penalizace)
    \end{itemize}

%------------------------------------------------

    \section{White box testování}

    Jeden z prvních testů celého program odhalil závažnou chybu v programu.
    Chyba byla odhalena až po detailním výpisu grafu aktuálně nejlepších řešení heruistiky.
    Teorie simulovaného ochlazování říka, že se snižující teplotou by se mělo řešení ustálit a zlepšovat.
    To na grafu zobrazeném na obrázku~\ref{fig:no-convergence} ale vůbec není patrné.

    Daný problém spočíval ve výpočtu a použití hodnoty delta (na obrázku~\ref{fig:main-loop}).
    Parametr delta se vypočítává odečtením hodnoty účelové funkce perturbovaného ohodnocení a hodnoty účelové funkce
    aktuálního ohodnocení.
    Následně je tato hodnota delta porovnána, zdali je větší než 0, což znamená zlepšení, v tom případě pertubované řešení
    nahrazuje aktuální ohodnocení a cyklus pokračuje dále.

    Problém nastával ale v~tom, že na parametru delta je také závislý výpočet pravděpodobnosti přijetí horšího řešení.
    V tomto případě je nutné počítat s~hodnotou delta v~absolutní hodnotě.
    Protože pro nízké teploty je tato delta velmi malá a zapříčinilo to přijetí mnoha zhoršujících řešení.
    Obrázek~\ref{fig:convergence} zobrazuje graf vývoje aktuálně nejlepšího řešení problému.
    Na tomto grafu je již vidět trend ochlazování a konvergence k optimálnímu řešení.

    \begin{figure}
        \centering
        \includegraphics[width=7cm]{images/no-convergence}
        \caption{Simulované ochlazování bez konvergence}
        \label{fig:no-convergence}
    \end{figure}

    \begin{figure}
        \centering
        \includegraphics[width=7cm]{images/convergence}
        \caption{Simulované ochlazování s konvergencí}
        \label{fig:convergence}
    \end{figure}

    Následně byly testovány jednotlivé parametry heuristiky a závislost jejich nastavení na úspěšnosti nalezení optimálního řešení.
    Na následujících grafech je zobrazana úspěšnost nalezení optimálního řešení.
    Heuristika nalézá také neoptimální řešení, která jsou blízko optimálnímu, tato chyba zde však porovnána nebyla.

    Základním testem bylo nastavení počáteční teploty.
    Obrázky~\ref{fig:initial_temperature_71}~a~\ref{fig:initial_temperature_218R} zobrazují úspěšnosti nalezení optimálního řešení
    na instancích wuf20-71R a wuf50-218R s různými hodnotami počáteční teploty.
    Testované hodnoty byly z množiny

    \begin{itemize}
        \item 500
        \item 1000
        \item 2000
        \item 3000
        \item 4000
        \item 5000
        \item 6000
    \end{itemize}

    V těchto testech se hodnota 5000 jakožto počáteční teplota ukázala být nejvhodnější s~přihlédnutím na úspěšnost
    v~složitejších instancích a délce výpočetního času.

    \begin{figure}
        \centering
        \includegraphics[width=7cm]{images/testing/initial_temperature/static_penalty_m5000/wuf20-71R}
        \caption{Experiment nastavení počáteční teploty 20-71R}
        \label{fig:initial_temperature_71}
    \end{figure}

    \begin{figure}
        \centering
        \includegraphics[width=7cm]{images/testing/initial_temperature/static_penalty_m5000/wuf50-218R}
        \caption{Experiment nastavení počáteční teploty 50-218R}
        \label{fig:initial_temperature_218R}
    \end{figure}

    Dále byl testován parametr chladícího faktoru.
    Tento parametr ovlivňuje rychlost konvergence metody a zároveň společně s nastavením počáteční a koncové teploty
    určuje počet kroků hlavního cyklu.

    Nastavení tohoto parametru bylo testováno na hodnotách

    \begin{itemize}
        \item .75
        \item .8
        \item .85
        \item .9
        \item .95
        \item .975
        \item .99
    \end{itemize}

    Grafy na obrázcích~\ref{fig:cooling_factor_71}~a~\ref{fig:cooling_factor_218R} zobrazují úspěšnost nalezení
    optimálního řešení v závislosti na nastavení hodnoty chladícího faktoru.

    \begin{figure}
        \centering
        \includegraphics[width=7cm]{images/testing/cooling_factor/wuf20-71R}
        \caption{Experiment nastavení chladícího faktoru 20-71R}
        \label{fig:cooling_factor_71}
    \end{figure}

    \begin{figure}
        \centering
        \includegraphics[width=7cm]{images/testing/cooling_factor/wuf50-218R}
        \caption{Experiment nastavení chladícího faktoru 50-218R}
        \label{fig:cooling_factor_218R}
    \end{figure}

    Experiment navrhuje jako nejlepší hodnotu .99, výsledná hodnota ale byla zvolena .975 z důvodu rychlejší konvergence - rychlejšího běhu algoritmu.

    Algoritmus obsahuje účelovou funkci, která nabývá hodnoty vah aktuálního ohodnocení, ale zároveň penalizuje nesplněné klauzule.
    Tímto postupem je zajištěna maximalizace vah řešení společně se snahou nalézt ohodnocení, které splňuje zadanou formuli.
    Zároveň je zde ale volnost, která umožuje heuristice některé klauzule nesplnit, pokud je splněná většina ostatních.

    Po několika iterací a manuálním testování úspěšnosti a efektivity algoritmu byla zvolena tato výpočetní formule (obrázek~\ref{fig:objective}),
    která úměrně penalizuje nesplněné klauzuje skalárním násobkem (penalizace).
    Tento skalární násobek je také parametr heuristiky a pokud je vyplněn 0, heuristika si automaticky vhodnou penalizaci
    dopočte z vektoru vah jednotlivých proměnných (obrázek~\ref{fig:auto_penalty}).

    \begin{figure}
        \centering
        \includegraphics[width=7cm]{images/objective}
        \caption{Účelová funkce heuristiky}
        \label{fig:objective}
    \end{figure}

    \begin{figure}
        \centering
        \includegraphics[width=7cm]{images/auto_penalty}
        \caption{Automatické vypočtení penalizačního skaláru}
        \label{fig:auto_penalty}
    \end{figure}

    Výsledný experiment pro nastavení správné hodnoty penalizace obsahoval tyto parametry

    \begin{itemize}
        \item -7000
        \item -5000
        \item -3000
        \item -1500
        \item -1000
        \item -500
        \item 0 (automatický výpočet dle vah)
    \end{itemize}

    Graf na obrázcích~\ref{fig:penalty_71}~a~\ref{fig:penalty_218} zobrazují procentuální úspěšnost nalezení optimálního řešení
    v~závislosti na penalizačním skaláru.

    \begin{figure}
        \centering
        \includegraphics[width=7cm]{images/testing/penalty/wuf20-71R}
        \caption{Experiment nastavení penalizačního skaláru 20-71R}
        \label{fig:penalty_71}
    \end{figure}

    \begin{figure}
        \centering
        \includegraphics[width=7cm]{images/testing/penalty/wuf50-218R}
        \caption{Experiment nastavení penalizačního skaláru 50-218R}
        \label{fig:penalty_218}
    \end{figure}

    Tento experiment potvrzuje dobré nastavení penalizačního merchanizmu a automatického výpočtu penalizace pomocí vektoru vah
    pro menší instance, které mají neoptimálně nastavené váhy (varianty Q a R).

    Automatický výpočet vah se ale neosvědčil při zpracovávní větších instancí a proto byla nakonec zvolena penalizační
    konstanta -5000, která měla obecně dobrou úspěšnost a zároveň zvládala neoptimální varianty úloh řešit s nadprůměrnou úspěšností.

%------------------------------------------------

    \section{Black box testování}

    Rámec black box testování již měl zafixované parametry heuristiky, které v minlé sekci vyšli jako nejlepší.
    Jedná se o počáteční teplotu 5000, odhlazující faktor 0.975 a fixní penalizaci -5000.

    V rámci této sekce byly zvoleny dvě hlavní skupiny instancí, na kterých bude úspěšnost heuristiky demonstrována

    \begin{itemize}
        \item wuf-20-71R
        \item wuf-50-218R
    \end{itemize}

    V rámci těchto skupin existují podskupiny M, N, Q a R, kde každá obsahuje 100 instancí.
    Jelikož heuristika využíva mnoho náhodných prvků, každá instance byla spuštěna 10.
    Celkově se tedy pro tyto dvě skupiny spustilo 8000 běhu algoritmu.

    Histogram na obrázku~\ref{fig:bar_weight_71} vyobrazuje procentuální odchylku váhy nalezeného ohodnocení
    od optimální váhy.
    V případě, že je odchylka menší než 1.0, heuristika nedokázala najít to nejlepší splňující ohodnocení.
    Pokud byla odchylka ostře větší než 1.0 znamená to, že heuristika nalezla ohodnocení s větším součtem vah.
    Jelikož máme data o optimálním řešení, říká nám to, že v těchto případech penalizace nesplněných klauzilí nebyla dostatečná
    a~heuristika zvolila jako výhodnější stavy s větším součtem vah.

    Pokud je odchylka přesně 1.0, tak heuristika našla optimální ohodnocení proměnných s největším váhovým součtem.

    \begin{figure}
        \centering
        \includegraphics[width=7cm]{images/testing/bar/weight/wuf20-71R}
        \caption{Experiment nalezení chyby v maximální váze 20-71R}
        \label{fig:bar_weight_71}
    \end{figure}

    \begin{figure}
        \centering
        \includegraphics[width=7cm]{images/testing/bar/weight/wuf50-218R}
        \caption{Experiment nalezení chyby v maximální váze 50-218R}
        \label{fig:bar_weight_218}
    \end{figure}

    V histogramech na obrázcích~\ref{fig:bar_sat_71}~a~\ref{fig:bar_sat_218} lze nahlédnout na procentuální počet splněných klauzulí.
    Na těchto grafech je vidět, že na instacích ze sady 20-71R dokázala heuristika téměř vždy nalézt splnitelné ohodnocení.
    Instance 50-218R mají o pár procent menší úspěšnost, hlavně varianta datové sady R.

    \begin{figure}
        \centering
        \includegraphics[width=7cm]{images/testing/bar/sat/wuf20-71R}
        \caption{Experiment nalezení chyby v maximální váze 20-71R}
        \label{fig:bar_sat_71}
    \end{figure}

    \begin{figure}
        \centering
        \includegraphics[width=7cm]{images/testing/bar/sat/wuf50-218R}
        \caption{Experiment nalezení chyby v maximální váze 50-218R}
        \label{fig:bar_sat_218}
    \end{figure}


%------------------------------------------------


    \section{Závěr}

    V této práci vznikla jednoduchá a lehce přepoužitelná knihovna pro souštění heuristiky simulovaného ochlazování na řešení
    NP-těžkého problém váženého SAT ohodnocení. Heuristika splňuje zadání práce a relativně úspěšně řeší i složitější varianty úloh.

    Implementací autor narazil na několik zajímavých poznatků a možných chyb, které při implementaci mohou nastat a opravil
    tyto chyby na základně experimentálního pozorování výsledků heuristiky.

    Zároveň proběhlo několik experimentálních pokusů pro určení ideálních parametrů heuristky, které následně byly v black-box
    fázi pozorovány za přínosné.

    Algoritmus do budoucích použití je vhodný spíše pro menší instance problémů, jelikož na instancích z sady 50-218R
    měl algoritmus úspěšnost nalezení optimálního řešení pod 50\%.

    Jelikož se jedná o randomizovaný algoritmus bylo provedeno několik běhů na stejných instancích a následně byly tyto výsledky
    započteny do finálních výsledků.

%----------------------------------------------------------------------------------------
%	REFERENCE LIST
%----------------------------------------------------------------------------------------

    %\clearpage

    \bibliography{literature}
    \bibliographystyle{plain}

%\begin{thebibliography}{99} % Bibliography - this is intentionally simple in this template
%
%\bibitem[Figueredo and Wolf, 2009]{Figueredo:2009dg}
%Figueredo, A.~J. and Wolf, P. S.~A. (2009).
%\newblock Assortative pairing and life history strategy - a cross-cultural
%  study.
%\newblock {\em Human Nature}, 20:317--330.
%
%\end{thebibliography}

%----------------------------------------------------------------------------------------

\end{document}
